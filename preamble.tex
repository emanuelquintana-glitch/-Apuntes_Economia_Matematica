% ============================================
% PREAMBLE.TEX - CONFIGURACIÓN GENERAL
% ============================================

% Codificación y fuentes
\usepackage[utf8]{inputenc}
\usepackage[T1]{fontenc}
\usepackage[spanish,es-tabla,es-nodecimaldot]{babel}
\usepackage{lmodern}
\usepackage{microtype}

% Matemáticas
\usepackage{amsmath,amssymb,amsthm,amsfonts}
\usepackage{mathtools}
\usepackage{mathrsfs}
\usepackage{bbm}
\usepackage{dsfont}

% Geometría y espaciado
\usepackage[a4paper,
            top=3cm,
            bottom=3cm,
            left=3.5cm,
            right=2.5cm,
            headheight=14pt]{geometry}
\usepackage{setspace}
\onehalfspacing

% Colores (Paleta azul estilo Göttingen)
\usepackage{xcolor}
\definecolor{goettingenblue}{RGB}{0,57,91}        % Azul principal
\definecolor{goettingenlightblue}{RGB}{0,120,180} % Azul claro
\definecolor{goettingendarkblue}{RGB}{0,40,70}    % Azul oscuro
\definecolor{accentblue}{RGB}{51,102,153}         % Azul acento
\definecolor{theoremblue}{RGB}{240,248,255}       % Fondo teoremas

% Gráficos y figuras
\usepackage{graphicx}
\usepackage{tikz}
\usetikzlibrary{arrows,shapes,positioning,calc,decorations.pathreplacing}
\usepackage{pgfplots}
\pgfplotsset{compat=1.18}

% Tablas
\usepackage{booktabs}
\usepackage{array}
\usepackage{multirow}
\usepackage{longtable}

% Enlaces e hipervínculos
\usepackage{hyperref}
\hypersetup{
    colorlinks=true,
    linkcolor=goettingenblue,
    citecolor=goettingenlightblue,
    urlcolor=accentblue,
    bookmarksnumbered=true,
    bookmarksopen=true,
    pdftitle={Apuntes de Economía Matemática},
    pdfauthor={Emanuel Quintana},
    pdfsubject={Economía Matemática},
    pdfkeywords={Economía, Matemáticas, Modelos Económicos}
}

% Bibliografía
\usepackage[backend=biber,
            style=authoryear,
            sorting=nyt,
            maxcitenames=2,
            maxbibnames=99]{biblatex}
\addbibresource{bibliography/referencias.bib}

% Teoremas y entornos matemáticos
\usepackage{thmtools}
\usepackage[framemethod=tikz]{mdframed}

\mdfdefinestyle{theoremstyle}{
    linecolor=goettingenblue,
    linewidth=2pt,
    topline=true,
    bottomline=true,
    rightline=false,
    leftline=false,
    innertopmargin=8pt,
    innerbottommargin=8pt,
    innerrightmargin=10pt,
    innerleftmargin=10pt,
    backgroundcolor=theoremblue,
    frametitlebackgroundcolor=goettingenblue,
    frametitlefont=\bfseries\color{white},
    frametitlerule=true,
    frametitlerulewidth=0pt,
}

\mdfdefinestyle{definitionstyle}{
    linecolor=goettingenlightblue,
    linewidth=2pt,
    topline=true,
    bottomline=true,
    rightline=false,
    leftline=false,
    innertopmargin=8pt,
    innerbottommargin=8pt,
    innerrightmargin=10pt,
    innerleftmargin=10pt,
    backgroundcolor=white,
    frametitlebackgroundcolor=goettingenlightblue,
    frametitlefont=\bfseries\color{white},
}

\mdfdefinestyle{examplestyle}{
    linecolor=accentblue,
    linewidth=1.5pt,
    topline=true,
    bottomline=true,
    rightline=false,
    leftline=true,
    innertopmargin=8pt,
    innerbottommargin=8pt,
    innerrightmargin=10pt,
    innerleftmargin=10pt,
    backgroundcolor=white,
}

% Definición de entornos
\declaretheoremstyle[
    mdframed={style=theoremstyle},
    headfont=\bfseries\color{goettingenblue},
    bodyfont=\normalfont,
    spaceabove=12pt,
    spacebelow=12pt,
]{theoremstyle}

\declaretheoremstyle[
    mdframed={style=definitionstyle},
    headfont=\bfseries\color{goettingenlightblue},
    bodyfont=\normalfont,
    spaceabove=12pt,
    spacebelow=12pt,
]{definitionstyle}

\declaretheoremstyle[
    mdframed={style=examplestyle},
    headfont=\bfseries\color{accentblue},
    bodyfont=\normalfont,
    spaceabove=10pt,
    spacebelow=10pt,
]{examplestyle}

\declaretheorem[style=theoremstyle,name=Teorema,numberwithin=chapter]{theorem}
\declaretheorem[style=theoremstyle,name=Lema,sibling=theorem]{lemma}
\declaretheorem[style=theoremstyle,name=Proposición,sibling=theorem]{proposition}
\declaretheorem[style=theoremstyle,name=Corolario,sibling=theorem]{corollary}

\declaretheorem[style=definitionstyle,name=Definición,numberwithin=chapter]{definition}
\declaretheorem[style=definitionstyle,name=Axioma,sibling=definition]{axiom}

\declaretheorem[style=examplestyle,name=Ejemplo,numberwithin=chapter]{example}
\declaretheorem[style=examplestyle,name=Ejercicio,sibling=example]{exercise}

\declaretheoremstyle[
    headfont=\bfseries\color{goettingendarkblue},
    bodyfont=\normalfont\itshape,
    spaceabove=8pt,
    spacebelow=8pt,
    qed=$\blacksquare$,
]{proofstyle}
\declaretheorem[style=proofstyle,name=Demostración,numbered=no]{prf}

% Encabezados y pies de página
\usepackage{fancyhdr}
\pagestyle{fancy}
\fancyhf{}
\fancyhead[LE]{\textcolor{goettingenblue}{\slshape\leftmark}}
\fancyhead[RO]{\textcolor{goettingenblue}{\slshape\rightmark}}
\fancyfoot[C]{\textcolor{goettingenblue}{\thepage}}
\renewcommand{\headrulewidth}{0.5pt}
\renewcommand{\headrule}{\hbox to\headwidth{\color{goettingenblue}\leaders\hrule height \headrulewidth\hfill}}

% Formato de capítulos
\usepackage{titlesec}
\titleformat{\chapter}[display]
  {\normalfont\huge\bfseries\color{goettingenblue}}
  {\filleft\Huge\textcolor{goettingenlightblue}{\chaptertitlename\ \thechapter}}
  {4ex}
  {\titlerule[2pt]\vspace{2ex}\filleft}
  [\vspace{2ex}\titlerule[2pt]]

\titleformat{\section}
  {\normalfont\Large\bfseries\color{goettingenblue}}
  {\thesection}{1em}{}

\titleformat{\subsection}
  {\normalfont\large\bfseries\color{goettingenlightblue}}
  {\thesubsection}{1em}{}

% Listas
\usepackage{enumitem}
\setlist[itemize]{label=\textcolor{goettingenblue}{$\bullet$}}
\setlist[enumerate]{label=\textcolor{goettingenblue}{\arabic*.}}

% Cajas de observaciones y notas
\usepackage[most]{tcolorbox}
\newtcolorbox{observacion}{
    colback=theoremblue,
    colframe=goettingenlightblue,
    fonttitle=\bfseries,
    title=Observación,
    arc=3mm,
    boxrule=1pt,
}

\newtcolorbox{nota}{
    colback=white,
    colframe=accentblue,
    fonttitle=\bfseries,
    title=Nota,
    arc=2mm,
    boxrule=1pt,
}

% Otros paquetes útiles
\usepackage{csquotes}
\usepackage{lipsum}
\usepackage{subcaption}